\documentclass{article}

\usepackage[utf8]{inputenc}
\usepackage[italian]{babel}

\usepackage{amsmath}
\usepackage{amssymb}

\usepackage{amsthm}
\theoremstyle{plain}
\newtheorem{thm}{Teorema}
\newtheorem{prop}{Proposizione}

\newcommand{\R}{\mathbb R}
\newcommand{\N}{\mathbb N}
\newcommand{\Rp}{\R_+}
\newcommand{\divides}{\mid}
\newcommand{\congr}{\equiv}
\newcommand{\abs}[1]{\left|#1\right|}
\newcommand{\norm}[1]{\|#1\|}
\newcommand{\tra}{^\top}
\newcommand{\chain}[1]{\{#1_n\}_{n\in\N}}

\usepackage[shortlabels]{enumitem}

\title{Metodi Probabilistici per l'Informatica\\[1ex]\large Riepilogo delle Dimostrazioni richieste}
\author{Alessandro Clerici}

\begin{document}
\maketitle




\section{Equivalenza tra matrici primitive e matrici irriducibili aperiodiche}


\subsection{Requisiti}
\begin{prop}\label{prop:rq}
	Sia $A\in\Rp^{k\times k}$ irriducibile e sia $d$ il suo periodo. Allora per ogni $i,j\in k$ esistono $r_{i,j},q_{i,j}\in\N$ tali che $0\le r_{i,j}<d\land 0<q_{i,j}$ e:
	\begin{enumerate}[(a)]
		\item \label{elem:rq1} $\forall s\in\N\qquad a^{(s)}_{i,j}>0\implies s\congr r_{i,j}\mod d$;
		\item \label{elem:rq2} $\forall n\ge q_{i,j}\qquad a^{(nd+r_{i,j})}_{i,j}>0$.
	\end{enumerate}
\end{prop}

\subsection{Teorema}
\begin{thm}
	Sia $A\in\Rp^{k\times k}$.
	$A$ è primitiva se e solo se $A$ è irriducibile e aperiodica.
\end{thm}
\begin{proof}
	$\Rightarrow)$
	Se $A$ è primitiva allora, per definizione, esiste $t\in\N$ tale che $A^t>0$.
	Naturalmente $A$ è irriducibile, essendo che per ogni $i,j\in k$ vale $a^t_{i,j}>0$.
	Inoltre dalla primitività segue che $A^{t+1}>0$. Dal momento che $A^t(i,i)>0$ e $A^{t+1}(i,i)>0$ esistono cicli di lunghezze $t$ e $t+1$ nel grafo rappresentato da $A$. Quindi se $d$ è il periodo di $A$ valgono $d\divides t$ e $d\divides t+1$, il che implica $d=1$.

	$\Leftarrow)$
	Se $A$ è irriducibile e il suo periodo è $d=1$, applicando la proposizione \ref{prop:rq} si ha che per ogni $i,j\in k$, $r_{i,j}=0$ e $a^{(n1+0)}_{i,j}>0$ per ogni $n\ge q_{i,j}$. Scegliendo $q:=\max_{i,j\in k}\{q_{i,j}\}$ si ha $A^q>0$. \qedhere
\end{proof}




\section{Modulo degli autovalori nelle matrici stocastiche}

\begin{thm}
	Sia $A\in\Rp^{k\times k}$ stocastica e $\mu$ autovalore di $A$. Allora $\abs{\mu}\le 1$.
\end{thm}
\begin{proof}
	\begin{align*}
		\abs{\mu}\cdot\norm{v} & = \abs{\mu}\sum_{j=1}^k\abs{v_j} = \sum_{j=1}^k\abs {\mu v_j}                  \\
		                       & = \sum_{j=1}^k\abs {(\mu v)_j} = \sum_{j=1}^k\abs {(v\tra A)_j}                \\
		                       & = \sum_{j=1}^k\abs {(v\tra A)_j} = \sum_{j=1}^k\abs {\sum_{i=1}^k v_i a_{i,j}} \\
		                       & = \sum_{j=1}^k\abs {(v\tra A)_j} = \sum_{j=1}^k \sum_{i=1}^k \abs{v_i} a_{i,j} \\
		                       & = \sum_{j=1}^k\abs {(v\tra A)_j} = \sum_{i=1}^k \abs{v_i} \sum_{j=1}^k a_{i,j} \\
		                       & = \sum_{i=1}^k \abs{v_i} = \norm{v}                                            \\
	\end{align*}
	Da cui, essendo $v\neq 0$:
	\begin{equation*}
		\abs{\mu}\leq 1 \qedhere
	\end{equation*}
\end{proof}




\section{Equivalenza tra stati ricorrenti e essenziali}


\subsection{Requisiti}
\begin{prop}\label{prop:sumtrans}
	In una catena di markov su $(S,\mu,P)$, sia $i\in S$ transiente. Allora, per ogni $i\in S$:
	\begin{equation*}
		\sum_{n\ge0}p^{(n)}(i,j)<+\infty
	\end{equation*}
\end{prop}


\subsection{Teorema}
\begin{thm}
	Sia $\chain X$ una catena di Markov definita su $(S,\mu,P)$. Sia $i\in S$.
	\begin{enumerate}[(a)]
		\item \label{elem:ricess1} se $i$ è ricorrente allora $i$ è essenziale;
		\item \label{elem:ricess2} se $i$ è essenziale e $S$ è finito, allora $i$ è ricorrente.
	\end{enumerate}
\end{thm}
\begin{proof}
	$\Rightarrow)$
	Per assurdo, sia $i$ non essenziale. Allora $i$ raggiunge uno stato $j$ fuori dalla sua classe in $m$ passi, quindi:
	\begin{equation*}
		\Pr_i(X_m=j)>0
	\end{equation*}
	Essendo $i$ ricorrente vale inoltre $\Pr(X_n=i \text{ per infiniti $n>0$})=1$.

	Ma i due eventi sono disgiunti, poiché una volta lasciata la classe di $i$ (per arrivare a $j$) questa non può più essere raggiunta. Quindi
	\begin{equation*}
		\Pr((X_n=i \text{ per infiniti $n>0$}),X_m=j) = 0
	\end{equation*}

	Ma al contempo, essendo il primo un evento certo, vale

	\begin{equation*}
		\Pr((X_n=i \text{ per infiniti $n>0$}),X_m=j) = \Pr(X_m=j)>0 \text,
	\end{equation*}
	assurdo.

	$\Leftarrow)$
	Sia $[i]$ la classe irriducibile che contiene $i$. Essendo $[i]$ una classe essenziale vale, per ogni $n\in\N$,
	\begin{equation*}
		\sum_{j\in[i]} P^{(n)}(i,j) = 1
	\end{equation*}
	Per assurdo, sia $i$ transiente. Allora in virtù della proposizione \ref{prop:sumtrans} vale, per ogni $j\in S$:
	\begin{equation*}
		\lim_{n\to+\infty} p^{(n)}(i,j) = 0 \text.
	\end{equation*}
	Ergo:
	\begin{align*}
		1 & = \sum_{j\in[i]} p^{(n)}(i,j)                    \\
		  & = \lim_{n\to+\infty} \sum_{j\in[i]} p^{(n)}(i,j) \\
		  & = \sum_{j\in[i]} \lim_{n\to+\infty} p^{(n)}(i,j) \\
		  & = \sum_{j\in[i]} 0 = 0 \text,
	\end{align*}
	assurdo. \qedhere
\end{proof}

\end{document}
