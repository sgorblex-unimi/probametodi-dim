%% Copyright (C) 2023 Alessandro Clerici Lorenzini
%
% This work may be distributed and/or modified under the
% conditions of the LaTeX Project Public License, either version 1.3
% of this license or (at your option) any later version.
% The latest version of this license is in
%   http://www.latex-project.org/lppl.txt
% and version 1.3 or later is part of all distributions of LaTeX
% version 2005/12/01 or later.
%
% This work has the LPPL maintenance status `maintained'.
%
% The Current Maintainer of this work is Alessandro Clerici Lorenzini
%
% This work consists of the files listed in work.txt


\section{Lemma di accoppiamento}


\subsection{Requisiti}
\begin{prop}\label{prop:neqvar}
	Siano $X\sim\mu$ e $Y\sim\pi$ variabili aleatorie su specificazioni nell'insieme finito $S$ (ossia siano $\mu$ e $\pi$ vettori stocastici rappresentanti le rispettive distribuzioni). Allora
	\begin{equation*}
		\dtv(\mu,\pi)\le\Pr(X\ne Y)
	\end{equation*}
\end{prop}


\subsection{Teorema}
\begin{lemma}[di accoppiamento]
	Sia $\{X_n,Y_n\}$ un accoppiamento su insieme degli stati $S$ e matrice di transizione $P$ primitiva. Se esistono $\epsilon>0$ e $n\in\N$ tali che per ogni $i,j\in S$
	\begin{equation*}
		\Pr(X_n\ne Y_n\mid X_0=i,Y_0=j) \le \epsilon \text,
	\end{equation*}
	allora
	\begin{equation*}
		\tau(\epsilon) \le n \text.
	\end{equation*}
\end{lemma}
\begin{proof}
	Sia $\pi$ la distribuzione stazionaria su $\chain X$ (o $\chain Y$).
	Fissato $i\in S$, sia $\chain{X'}$ la catena su $(S,\delta_i,P)$, dove $\delta_i$ è la distribuzione che ha probabilità $1$ di partire in $i$, e $\chain{Y'}$ la catena su $(S,\pi,P)$. Allora $p\ape{n}_i=\delta_i P^n$ è la distribuzione di $X'_n$ e $\pi$ quella di $Y'_n$.
	Si ha:
	\begin{align*}
		\Pr(X'_n\ne Y'_n) & = \Pr(X_n\ne Y'_n\mid X_0=i)                                                          \\
		                  & = \sum_{j\in S} \Pr(X_n\ne Y'_n\mid X_0=i,Y'_0=j)\Pr(Y'_0=j)                          \\
		                  & = \sum_{j\in S} \underbrace{\Pr(X_n\ne Y_n\mid X_0=i,Y_0=j)}_{\le\epsilon}\Pr(Y'_0=j) \\
		                  & \le \epsilon \sum_{j\in S} \pi_j = \epsilon
	\end{align*}
	Da cui, per la proposizione \ref{prop:neqvar}:
	\begin{align*}
		\dtv(p\ape{n}_i,\pi) & \le \epsilon
	\end{align*}
	e quindi $\tau(\epsilon)\le n$.
\end{proof}
