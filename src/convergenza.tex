%% Copyright (C) 2023 Alessandro Clerici Lorenzini
%
% This work may be distributed and/or modified under the
% conditions of the LaTeX Project Public License, either version 1.3
% of this license or (at your option) any later version.
% The latest version of this license is in
%   http://www.latex-project.org/lppl.txt
% and version 1.3 or later is part of all distributions of LaTeX
% version 2005/12/01 or later.
%
% This work has the LPPL maintenance status `maintained'.
%
% The Current Maintainer of this work is Alessandro Clerici Lorenzini
%
% This work consists of the files listed in work.txt


\subsection{Requisiti}
\begin{lemma}[di contrazione]\label{lemma:contra}
	Sia $P\in[0,1]^{k\times k}$ stocastica e $u,v\in[0,1]^k$ stocastici. Allora
	\begin{equation*}
		\norm{u\tra P-v\tra P} \le \beta(P)\norm{u-v}
	\end{equation*}
	dove $\beta(P)$ è il coefficiente ergodico di $P$.
\end{lemma}


\subsection{Teorema}
\section{Velocità di convergenza minima}
\begin{thm}
	In una catena primitiva sulla matrice $P$ tale che $P^t>0$, per ogni $\epsilon>0$ vale
	\begin{equation*}
		\tau(\epsilon)\le t\left(\frac{\log\epsilon}{\log\beta(P^t)}+1\right)
	\end{equation*}
\end{thm}
\begin{proof}
	Sia $\chain X$ la catena e $\pi$ la sua distribuzione stazionaria. Sia $S$ il suo insieme degli stati e dato $i\in S$ sia $\delta_i$ la distribuzione che assegna a $i$ probabilità $1$. Fissato $n\ge t$, sia $n=qt+r$. Si noti che $\beta(P^t)<1$ in quanto la matrice non ha colonne di zeri.
	\begin{align*}
		\dtv(\delta_i\tra P^n,\pi\tra) & = \frac12\norm{\delta_i\tra P^n-\pi\tra}                                   \\[1ex]
		                               & = \frac12\norm{\delta_i\tra P^r\cdot (P^t)^q-\pi\tra}                      \\[1ex]
		                               & \le \frac12\beta(P^t)^q\underbrace{\norm{\delta_i\tra P^r-\pi\tra}}_{\le2} \\
		                               & \le \beta(P^t)^{\frac{n-r}{t}} \le \beta(P^t)^{\frac n t -1}
	\end{align*}
	Fissato $\epsilon>0$:
	\begin{align*}
		\beta(P^t)^{\frac n t -1}             & \le \epsilon                                            \\
		\left(\frac nt-1\right)\log\beta(P^t) & \le \log\epsilon                                        \\
		\frac nt-1                            & \ge \frac{\log\epsilon}{\log\beta(P^t)}                 \\
		n                                     & \ge t\left(\frac{\log\epsilon}{\log\beta(P^t)}+1\right)
	\end{align*}
	poiché ciò vale per ogni $i\in S$, ne segue la tesi.
\end{proof}
