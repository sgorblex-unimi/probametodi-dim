\section{Teorema di ergodicità}


\subsection{Requisiti}
\begin{prop}\label{prop:stocconv}
	Sia $P\in[0,1]^{k\times k}$ stocastica primitiva. Allora esiste $\epsilon$ tale che $0\le\epsilon<1$ e per ogni coppia $u,v\in[0,1]^k$ di vettori stocastici:
	\begin{equation*}
		\norm{v\tra P^n-u\tra P^n} = O(\epsilon^n)
	\end{equation*}
\end{prop}


\subsection{Teorema}

\begin{thm}
	Sia $\chain{X}$ una catena primitiva con insieme di stati $S=\{1,\dots,k\}$. Allora
	\begin{enumerate}[(a)]
		\item \label{elem:erg1} $\chain{X}$ possiede una e una sola distribuzione stazionaria $\pi$;
		\item \label{elem:erg2} $\pi=\left\{ \frac{1}{\ev_i(\tau_i)} \right\}_{i\in S}$;
		\item \label{elem:erg3} $\chain{X}$ è ergodica e ha distribuzione limite $\pi$.
	\end{enumerate}
\end{thm}
\begin{proof}
	Sia $P$ la matrice di transizione di $\chain{X}$.

	\ref{elem:erg1} Siano $\pi,v$ distribuzioni stazionarie per $\chain{X}$ (ne esiste sempre almeno una). Per la proposizione \ref{prop:stocconv} vale
	\begin{align*}
		\norm{v\tra P^n-\pi\tra P^n} & \to 0        \\
		\norm{v\tra-\pi\tra}         & \to 0 \text,
	\end{align*}
	possibile solo se $v=\pi$.

	\ref{elem:erg2} Sia $\pi\ape{i}$ la distribuzione stazionaria costruita sullo stato ricorrente $i$. Poiché la distribuzione stazionaria è unica si ha, per ogni $i,j\in S$:
	\begin{equation*}
		\pi\ape{j}_i = \pi\ape{i}_i = \frac{1}{\ev_i(\tau_i)}
	\end{equation*}

	\ref{elem:erg3} Fissato $j\in S$, sia $\delta_j$ la distribuzione iniziale che associa a $j$ probabilità $1$. Per la proposizione \ref{prop:stocconv} vale
	\begin{align*}
		\norm{\delta_j\tra P^n-\pi}                   & \to 0 \\
		\sum_{i\in S}\abs{(\delta_j\tra P^n)_i-\pi_i} & \to 0
	\end{align*}
	Essendo i termini della somma non negativi, questo è possibile solo se, per ogni $i\in S$:
	\begin{equation*}
		(\delta_j\tra P^n)_i \to \pi_i
	\end{equation*}
	Ovvero $\Pr_j(X_n=i)\to \pi_i$. \qedhere
\end{proof}
