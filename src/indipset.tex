%% Copyright (C) 2023 Alessandro Clerici Lorenzini
%
% This work may be distributed and/or modified under the
% conditions of the LaTeX Project Public License, either version 1.3
% of this license or (at your option) any later version.
% The latest version of this license is in
%   http://www.latex-project.org/lppl.txt
% and version 1.3 or later is part of all distributions of LaTeX
% version 2005/12/01 or later.
%
% This work has the LPPL maintenance status `maintained'.
%
% The Current Maintainer of this work is Alessandro Clerici Lorenzini
%
% This work consists of the files listed in work.txt


\section{Correttezza del MCMC per indipendent set}

Sia $G=(V,E)$ un grafo non orientato, con $E\ne\emptyset$, e sia $S$ l'insieme di insiemi indipendenti su $G$, ossia
\begin{equation*}
	S := \{A\subseteq V\mid \forall u,v\in A, \{u,v\}\notin A\} \text.
\end{equation*}
Sia $Z_G:=\card S$ e $\pi:=\{\frac{1}{Z_G}\}_{i\in S}$. Si consideri l'algoritmo \ref{alg:indset}.

\begin{algorithm}
	\DontPrintSemicolon
	\KwData{$G=(V,E)$,$n$}

	$A\asn\{\UR{V}\}$

	\For{$n$ volte}{
		$V\asn\UR{V}$ \;
		\If{$V\in A$}{
			$A\asn A\setminus\{v\}$ \;
		}\ElseIf{$\forall z\in A\quad \{z,v\}\notin E$}{
			$A\asn A\cup\{v\}$ \;
		}
	}
	\Return{$A$} \;

	\caption{Algoritmo per generazione casuale di indipendent set.}
	\label{alg:indset}
\end{algorithm}

L'algoritmo simula i primi $n$ passi della catena di Markov $\chain X$ costruita sugli stati $S$ e tabella di transizione $P$ definita come segue (denotiamo con $A\div B$ la differenza simmetrica di $A$ e $B$):
\begin{equation*}
	P(A,B):=\begin{cases}
		0                  & \qquad \card{A\div B}>1 \\
		\frac 1 k          & \qquad \card{A\div B}=1 \\
		1 - \frac{d(A)}{k} & \qquad A=B
	\end{cases}
\end{equation*}
Dove $d_A$ è il numero di indipendent set vicini di $A$, ossia
\begin{equation*}
	d_A:=\card{\{C\in S\mid \card{A\div C}=1\}} \text.
\end{equation*}
Si noti che $d_A\le k$, in quanto ogni nodo determina al più un vicino indipendent set.

\begin{thm}
	La catena $\chain X$ è un algoritmo MCMC corretto per la generazione uniforme di indipendent set sul grafo $G$.
\end{thm}
\begin{proof}
	La matrice $P$ è irriducibile, infatti è sempre possibile passare da un indipendent set $A$ a uno $B$ rimuovendo prima gli elementi in $A\setminus B$ e aggiungendo poi quelli in $B\setminus A$ (viene mantenuto lo status di indipendent set in quanto sottoinsiemi di $A$ e $B$ sono anch'essi indipendent set).

	La matrice è anche aperiodica, infatti sia $\{u,v\}\in E$ (si ricorda che $E\ne\emptyset$). Allora $d_{\{u\}}<k$, in quanto l'insieme $\{u,v\}$ non è indipendente. Ergo $P(\{u\},\{u\})=1-\frac{d_{\{u\}}}{k}>0$, quindi il grafo associato alla catena possiede un cappio.

	$\chain X$ è quindi primitiva ed ergodica e possiede un'unica distribuzione stazionaria, nonché distribuzione limite. Dimostriamo che tale distribuzione è $\pi$ dimostrando che essa è reversibile per $\chain X$. Per i casi $A=B$ e $\card{A\div B}>1$ la reversibilità è ovvia. Per il caso $\card{A\div B}=1$, essendo $\pi$ uniforme la reversibilità segue dalla simmetria di $P$, dovuta alla commutatività dell'operazione di differenza simmetrica.
\end{proof}
