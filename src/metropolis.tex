\section{Correttezza dell'algoritmo di Metropolis}

Sia $\{\pi\}_{i\in S}$ una distribuzione arbitraria su $S$. L'algoritmo di Metropolis costruisce un grafo non orientato connesso sui nodi in $S$ e, simulando una passeggiata aleatoria sul grafo, genera un elemento di $S$ con distribuzione $\pi$.
Costruito il grafo $G=(S,E)$, l'algoritmo implementa la catena $\chain X$ definita sulla tripla $(S,\{\card S^{-1}\}_{i\in S},P)$, con $P$ definita come segue:
\begin{equation*}
	P(i,j):=\begin{cases}
		0                                                                                                    & \qquad i\ne j\land \{i,j\}\notin E \\[1ex]
		\displaystyle \frac{1}{d_i} \min\left\{\frac{\pi_j d_i}{\pi_i d_j},1\right\}                         & \qquad \{i,j\}\in E                \\[2ex]
		\displaystyle 1 - \frac{1}{d_i} \sum_{\{i,l\}\in E} \min\left\{\frac{\pi_l d_i}{\pi_i d_l},1\right\} & \qquad i=j
	\end{cases}
\end{equation*}

La catena è implementata tramite l'algoritmo \ref{alg:metropolis}.
\begin{algorithm}
	\DontPrintSemicolon
	\KwData{$\pi$,$G=(S,E)$,$n$}

	$i\asn\UR{S}$ \;

	\For{$n$ volte}{
		$j\asn\UR{$\{k\in S\mid \{j,k\}\in E\}$}$ \;
		\If{$\pi_i d_j\le\pi_j d_i$}{
			$i\asn j$ \;
		}\Else{
			$i\asn j$ con probabilità $\dfrac{\pi_j d_i}{\pi_i d_j}$ \;
		}
	}
	\Return{$i$} \;

	\caption{Algoritmo di Metropolis}
	\label{alg:metropolis}
\end{algorithm}

\begin{thm}
	La catena $\chain X$ è un algoritmo MCMC corretto per la distribuzione $\pi$.
\end{thm}
\begin{proof}
	La matrice $P$ è irriducibile in quanto $G$ è connesso per ipotesi e gli archi presenti in $G$ sono presenti anche nel grafo associato a $\chain X$.

	La matrice è anche aperiodica se $G$ presenta un ciclo di lunghezza dispari oppure se il grafo associato alla catena ha un cappio, ossia se per qualche $i,j$ vale $\pi_j d_i<\pi_i d_j$ e quindi
	\begin{equation*}
		1 - \frac{1}{d_i} \sum_{\{i,l\}\in E} > 1 - \frac{d_i}{d_i} > 0
	\end{equation*}

	Quanto alla stazionarietà di $\pi$, è sufficiente provare che $\pi$ è reversibile. Naturalmente per $i=j$ o $P(i,j)=0$ l'uguaglianza è ovvia, mentre per $\{i,j\}\in E$ supponiamo $\pi_j d_i \le \pi_i d_j$:
	\begin{equation*}
		\pi_i P(i,j) = \pi_i \frac{1}{d_i} \frac{\pi_j d_i}{\pi_i d_j} = \pi_j \frac{1}{d_j} = \pi_j P(j,i) \qedhere
	\end{equation*}
\end{proof}
