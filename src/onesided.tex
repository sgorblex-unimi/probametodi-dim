% TODO: sistemare un po'. Includere risultati sul tempo di esecuzione.
\section{Riduzione della probabilità di errore negli algoritmi 1-sided error}
\newcommand{\Aalg}{\mathcal A}

Sia $\Aalg$ un algoritmo probabilistico 1-sided error per la funzione di decisione $q:I\to\{0,1\}$.
Dato $k\in\N_+$, sia $\Aalg_k$ l'algoritmo \ref{alg:1sided}, che ripete $\Aalg$ $k$ volte a meno che produca un output positivo.

\begin{algorithm}
	\DontPrintSemicolon
	\KwData{$x$}
	\SetKwData{Out}{out}

	$i\asn1$ \;
	\While{$\Out=0\land i\le k$}{
		$\Out\asn\Aalg(x)$ \;
		$i\asn i+1$ \;
	}
	\Return{\Out} \;

	\caption{Algoritmo $\Aalg_k$ per la riduzione della probabilità di errore di $\Aalg$.}
	\label{alg:1sided}
\end{algorithm}

\begin{thm}
	La probabilità che $\Aalg_k$ sbagli è al più $2^{-k}$.
\end{thm}
\begin{proof}
	Per definizione di algoritmo 1-sided error si ha $\Pr(\Aalg=0\mid q(x)=1)\le\frac 12$. Per $\Aalg_k$:
	\begin{equation*}
		\Pr(\text{errore}) = \Pr(\Aalg_k=0\mid q(x)=1) = \Pr(\Aalg=0\mid q(x)=1)^k \le \left(\frac{1}{2}\right)^k \text. \qedhere
	\end{equation*}
\end{proof}
