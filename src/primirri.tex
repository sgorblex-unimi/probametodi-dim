%% Copyright (C) 2023 Alessandro Clerici Lorenzini
%
% This work may be distributed and/or modified under the
% conditions of the LaTeX Project Public License, either version 1.3
% of this license or (at your option) any later version.
% The latest version of this license is in
%   http://www.latex-project.org/lppl.txt
% and version 1.3 or later is part of all distributions of LaTeX
% version 2005/12/01 or later.
%
% This work has the LPPL maintenance status `maintained'.
%
% The Current Maintainer of this work is Alessandro Clerici Lorenzini
%
% This work consists of the files listed in work.txt


\section{Equivalenza tra matrici primitive e matrici irriducibili aperiodiche}


\subsection{Requisiti}
\begin{prop}\label{prop:rq}
	Sia $A\in\Rp^{k\times k}$ irriducibile e sia $d$ il suo periodo. Allora per ogni $i,j\in k$ esistono $r_{i,j},q_{i,j}\in\N$ tali che $0\le r_{i,j}<d\land 0<q_{i,j}$ e:
	\begin{enumerate}[(a)]
		\item \label{elem:rq1} $\forall s\in\N\qquad a\ape{s}_{i,j}>0\implies s\congr r_{i,j}\mod d$;
		\item \label{elem:rq2} $\forall n\ge q_{i,j}\qquad a\ape{nd+r_{i,j}}_{i,j}>0$.
	\end{enumerate}
\end{prop}

\subsection{Teorema}
\begin{thm}
	Sia $A\in\Rp^{k\times k}$.
	$A$ è primitiva se e solo se $A$ è irriducibile e aperiodica.
\end{thm}
\begin{proof}
	$\Rightarrow)$
	Se $A$ è primitiva allora, per definizione, esiste $t\in\N$ tale che $A^t>0$.
	Naturalmente $A$ è irriducibile, essendo che per ogni $i,j\in k$ vale $a\ape{t}_{i,j}>0$.
	Inoltre dalla primitività segue che $A^{t+1}>0$. Dal momento che $A^t(i,i)>0$ e $A^{t+1}(i,i)>0$ esistono cicli di lunghezze $t$ e $t+1$ nel grafo rappresentato da $A$. Quindi se $d$ è il periodo di $A$ valgono $d\divides t$ e $d\divides t+1$, il che implica $d=1$.

	$\Leftarrow)$
	Se $A$ è irriducibile e il suo periodo è $d=1$, applicando la proposizione \ref{prop:rq} si ha che per ogni $i,j\in k$, $r_{i,j}=0$ e $a\ape{n1+0}_{i,j}>0$ per ogni $n\ge q_{i,j}$. Scegliendo $q:=\max_{i,j\in k}\{q_{i,j}\}$ si ha $A^q>0$. \qedhere
\end{proof}
