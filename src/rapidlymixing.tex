\section{Esempio di catena rapidly mixing}
Si consideri l'insieme $S$ di possibili permutazioni di un mazzo di $k$ carte e sia $\chain X$ la catena definita sugli stati $S$ dove una transizione è definita dallo scegliere una carta uniformemente a caso e porla in cima al mazzo.

Definiamo l'accoppiamento $\{X_n,Y_n\}_{n\in\N}$, dove $Y_n$ è ottenuto da $Y_{n-1}$ scegliendo la stessa carta di $X_{n-1}$ e ponendola in cima al mazzo (invece di scegliere casualmente).

\begin{thm}
	La catena $\chain X$ è rapidly mixing.
\end{thm}
\begin{proof}
	Se $X_n\ne Y_n$ allora, a prescindere dallo stato iniziale, esiste una carta nel mazzo che non è stata scelta nei primi $n$ passi. Quindi per ogni $i,j\in S$:
	\begin{equation*}
		\Pr(X_n\ne Y_n\mid X_0=i,Y_0=j) \le k\left(1-\frac 1k\right)^n \le ke^{-\frac nk} \text,
	\end{equation*}
	l'ultima ottenuta in virtù della disuguaglianza $(1-x)\le e^{-x}$ per $x=\frac 1k$.
	Scelto $\epsilon>0$, si ha $ke^{-\frac nk}$ per $n\ge k\log k+k\log\frac{1}{\epsilon}$. Per il lemma di accoppiamento:
	\begin{equation*}
		\tau(\epsilon)\le k\left(\log k+\log\frac{1}{\epsilon}\right) \text. \qedhere
	\end{equation*}
\end{proof}
